% Checked with grammarly
\section{Validation of Results}
\label{sec:validation}
% To check results after generation such statistics as min, max, mean, variance, skew are compared. Additionally, missing values,  distinct values and outliers are checked. 
After the generation, the new dataset is validated to analyze and check if statistics are preserved. 
The data and error characteristics of the freshly generated dataset are compared against characteristics of the original dirty dataset.
The following soft constraints are used to validate the generated dataset:
\begin{itemize}
    \item The mean difference of the generated and the original dirty datasets should be within 10\%.
    \item The generated dataset's quantiles should be within 
    \item[~]$[Q_{dirty}(0.25) - 1.5\cdot IQR, Q_{dirty}(0.75) + 1.5\cdot IQR]$.
    \item Min and max are compared, and should be equivalent.
    \item The number of distinct values in generated dataset and estimated number~\ref{eq:distincts} should be within 5\%.
    \item The number of generated missing values, outliers, typos, replacements, and swaps differentiates by maximum  of 10\% from the original dirty numbers multiplied with the scaling factor.
    \item The number of unique values for each error is compared to the estimated, similar to the number of distinct items above.
\end{itemize}

\begin{equation}
\label{eq:distincts}
d_{generated} = d_{clean} + d_{mv} + d_{typos} + d_{outliers} + d_{swaps}
\end{equation}
If any of constraints are violated, the user is informed about these specific violations. 
Additionally, characteristics of the generated dataset are stored into separate logging files. 
They contain information such as statistics, error distribution, and binary masks.
All this information can be useful since recomputing it can be time consuming, especially at large scales.

\textbf{Local validation:} To compute statistical properties of the freshly generated data, Pandas API is used.

\textbf{Distributed validation:} 
In distributed data generation, Spark SQL aggregation functions are used to accumulate statistics.
Interestingly, the performance of aggregating statistics impacts execution time. 
Thus, Spark approximate algorithms were tried out, such as \textsc{approx\_count\_distinct}.
But for small scaling factors these were inaccurate even with 10\% precision, estimating the number of distinct items less in original dirty dataset.
Consequently, estimations and in-expectation statistics are not sufficient for this framework.
Error distribution is computed while error generation.