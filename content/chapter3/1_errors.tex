% Checked with grammarly
\section{Types of Errors}
\label{sec:error_types}

Data errors are values that differ from the ground truth.
In this context, we differentiate between data with and without errors by calling it clean or dirty. 
Completely clean data is ground truth (GT).  


In the context of this benchmark, we have a clean ground-truth dataset and a dirty version with introduced errors.  
Leveraging this, faulty values can be found by locating the cells that differ between dirty and clean. 
This information is helpful to detect errors and their distribution, and frequencies of specific values being error prone.  
Moreover, it is useful for classification of error types.


\textbf{Five error types}, that are common in the real-world datasets, are considered in this benchmark: 
Missing values, typos, outliers, replacements and swaps. 
The error types and descriptions are shown in Table~\ref{tab:generator_errors}.


\textbf{Missing value} is the simplest type of error that is frequently seen in our experimentation. 
There are three sub categories of missing values: 
Missing completely at random (MCAR), missing at random (MAR), missing not at random (MNAR). 
MCAR is a value that is missing completely independently. 
MAR occurs when the missingness is not random, but where it can be fully accounted for by variables where there is complete information. 
MNAR means that the missingness of an observation depends on its values. MNAR analysis is problematic because the distribution of the missing observations depends on both observed values and unobserved values.

% TODO fix
Missing data present various problems. First, the absence of data reduces statistical power, which refers to the probability that the test will reject the null hypothesis when it is false. Second, the lost data can cause bias in the estimation of parameters. Third, it can reduce the representativeness of the samples.

The benchmark supports MCAR and MAR.


\textbf{Typo}


\textbf{Outlier}

\textbf{Replacement} is a flipped value that was chosen from the existing set of valid value. For example, in feature with distinct set of values \emph{\{A, B, C\}}, \emph{A} can be replaced by \emph{D}. 


\textbf{Swap} is a pair of values swapped between two columns: numerical and categorical, categorical and categorical or two categorical.





The focus is on a single source datasets, thus multiple data source datasets and errors that occur during schema integration are not considered.


Data errors can be categorized by levels of their appearance in the data: single value, within a single feature, within a single tuple, within several features or tuples.

% Our error generation includes properties such as:
% \begin{itemize}
%     \item Error type: A variety of different error types can be introduced to the data.
%     \item Error rate: A configurable for each error rate, that specifies am amount of occurances of respective error.
%     \item Reproducibility: Each and every introduced error can be reproduced and tracked.
% \end{itemize}


% \begin{table}[!h]
% \caption{\label{tab:generator_errors}Errors preserved by generator}
% \begin{tabular}{l|l|l}
% \toprule
% Error name    & Parameters                                                                                                             & Description                                                                           \\ 
% \midrule
% Missing value & \begin{tabular}[c]{@{}l@{}}text text text text text text \\ text text text text text text\end{tabular} & \begin{tabular}[c]{@{}l@{}}text text text text text text \\ text text text text text text\end{tabular} \\
% Typo          &\begin{tabular}[c]{@{}l@{}}text text text text text text \\ text text text text text text\end{tabular} & \begin{tabular}[c]{@{}l@{}}text text text text text text \\ text text text text text text\end{tabular} \\
% Outlier       &\begin{tabular}[c]{@{}l@{}}text text text text text text \\ text text text text text text\end{tabular} & \begin{tabular}[c]{@{}l@{}}text text text text text text \\ text text text text text text\end{tabular} \\
% Replacement   &\begin{tabular}[c]{@{}l@{}}text text text text text text \\ text text text text text text\end{tabular} & \begin{tabular}[c]{@{}l@{}}text text text text text text \\ text text text text text text\end{tabular} \\
% Swap          &\begin{tabular}[c]{@{}l@{}}text text text text text text \\ text text text text text text\end{tabular} & \begin{tabular}[c]{@{}l@{}}text text text text text text \\ text text text text text text\end{tabular} \\
% \bottomrule
% \end{tabular}
% \end{table}