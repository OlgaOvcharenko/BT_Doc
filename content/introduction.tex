\chapter{Introduction}
% Context (1)
Data integration, cleaning and preparation are a part of the Data Science Lifecycle. It is an exploratory process that takes a lot of effort, time and experiments.
Data engineers and data scientists in general spend between 80\% and 90\% of their time just on finding, integrating and cleaning datasets~\cite{80cleansurvey, dataintegration80}.
\\\\
Database (DB) an Machine Learning (ML) communities have been working on problems associated with the dirty data~\cite{cleanml}. 
The ML community concentrates on impact of dirty data and noise on ML models~\cite{cleanml}. It was studied that noise can have negligible effect on accuracy~\cite{processingsys, outperformstudy}, but ML models can also be sensitive to the dirty data, especially dirty labels~\cite{classificationnoisesurvey}. The DB community's focus is on error detection and repairing ~\cite{Hellerstein08quantitativedata, duplicatesstudy}, but not on how dirty data influences the quality of ML models. 
\\\\
Performance of ML models and data preparation are interconnected. Data cleaning is part of the ML pipeline that aims to automate machine learning workflow. 
ML pipeline consists of several steps: data collection, data cleaning, feature extraction, model training, model evaluation, model visualization, and model serving.
Data preparation is a key step and obeys the garbage in, garbage out principle (GIGO). GIGO is the concept that flawed input data produces nonsense output.
\\\\
% Problems (1-3) and Existing work (1)
Data cleaning is the process of removing faulty values from a dataset. It consists of two steps: error detection and repair. Error detection aims to identify errors, while error repair imputes/fixes flawed values using knowledge gathered from the input data. 
There are many existing techniques~\cite{duplicatesstudy, tdeexcel} and frameworks to clean the data such as HoloDetect/HoloClean~\cite{holoclean, holodetect}, Raha/Baran~\cite{raha, baran}, BoostClean~\cite{boostclean}, AlphaClean~\cite{alphaclean}, ActiveClean~\cite{activeclean}. 
Existing techniques and frameworks are limited by additional user input such as hand-crafted constraints~\cite{bart}, or user manual labeling~\cite{raha, baran}. For evaluation of data cleaning approaches the ground truth (GT) is required that is not always available and is time-consuming to collect. Sufficient data cleaning often requires an extra user input such as hand-crafted constraints~\cite{bart}, or user manual labeling.
Additionally, many datasets are private/confidential that means that they can not be accessed, no source is provided, or the ground truth is missing. 
% Idea (1) and Contributions (1) TODO Rewrite
% TODO list of contributions
\\\\
Our contribution is to create a tool for the distributed data and error generation, that can be used in ML and DB communities. 
Currently, existing work focuses on certain error and repairs, but not the general evaluation of the approaches.
\\\\
We introduce a data cleaning for ML benchmark, that is designed to solve the problem of evaluation of data cleaning approaches, and the problem of the missing ground truth.
WashHouse scales clean and dirty versions of the dataset, preserving data and error distributions. 
WashHouse extracts real error patterns from the dataset and scales data with configurable errors and error rates. 
It covers eight real-world error classes (e.g., missing values, outliers).

% TODO Add contributions
Generate datasets with scalable error distributions. Ability to preserve mean, distincts etc. Generate large datasets efficiently.

% TODO Is structure needed? 